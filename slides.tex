\documentclass{beamer}

\usetheme{focus}
% Add option [numbering=none] to disable the footer progress bar
% Add option [numbering=fullbar] to show the footer progress bar as always full with a slide countx


\definecolor{main}{RGB}{55, 135, 177}
\definecolor{text}{RGB}{15, 40, 60}
\definecolor{background}{RGB}{255, 255, 255}

\setbeamertemplate{block begin}
{
  \par\vskip\medskipamount%
  \IfStrEq{\insertblocktitle}{}{}{
      \begin{beamercolorbox}[colsep*=.75ex]{block title}
        \usebeamerfont*{block title}\insertblocktitle%
      \end{beamercolorbox}%
  }
  {\parskip0pt\par}%
  \ifbeamercolorempty[bg]{block title}
  {}
  {\ifbeamercolorempty[bg]{block body}{}{\nointerlineskip\vskip-0.5pt}}%
  \usebeamerfont{block body}%
  \begin{beamercolorbox}[colsep*=.75ex,vmode]{block body}%
    \ifbeamercolorempty[bg]{block body}{\vskip-.25ex}{\vskip-.75ex}\vbox{}%
}


\setbeamercovered{transparent}

%------------------------------------------------

\usepackage{booktabs} % Required for better table rules
\usepackage{xstring}
\usepackage{tikz}
\renewcommand{\baselinestretch}{1.3}
\usepackage[T1]{fontenc}
\usepackage[sfdefault]{AlegreyaSans}
\usepackage{bussproofs}
\EnableBpAbbreviations
\usepackage{multicol}


\title{Type Theory and $\lambda$-calculus}

\author{Alireza Mahmoudian}

\titlegraphic{\includegraphics[scale=.2]{res/ut-logo.png}}

\institute{School of Mathematics,\\ Statistics and Computer Science\\ University of Tehran}

\date{May 2024}


\begin{document}


\begin{frame}
	\maketitle
\end{frame}

\def\secProg{A Very Simple Programming Language}
\def\subDec{Short on Declarative Programming}
\def\subLambda{Untyped $\lambda$-calculus}
\def\secType{Enter the Type}
\def\subSTLC{Introducing Simple Types}
\def\subTypability{Typability}
\def\secTruth{Back to Logic: Truth and Proof}
\def\subTruth{What is called \emph{truth}?}
\def\subBHK{The BHK Interpretation}
\def\subIntLog{Intuitionistic Propositional Logic}
\def\secDep{Beyond $0$-order Logic}
\def\subDepFun{Dependent Function Type}

\begin{frame}{Agenda}
	\begin{enumerate}
		\item \secProg
		\begin{enumerate}
			\item \subDec
			\item \subLambda
		\end{enumerate}
		\item \secType
		\begin{enumerate}
			\item \subSTLC
			\item \subTypability
		\end{enumerate}
		\item \secTruth
		\begin{enumerate}
			\item \subTruth
			\item \subBHK
			\item \subIntLog
		\end{enumerate}
		\item \secDep
		\begin{enumerate}
			\item \subDepFun
		\end{enumerate}
	\end{enumerate}
\end{frame}

\section{A Very Simple\\ Programming Language}

\begin{frame}{\subDec}
	Imperative programming language (like C, Java, Python \dots)
	\begin{itemize}
		\pause
		\item Executes a sequence of commands
		\pause
		\item Transforming the state of the machine
	\end{itemize}
	
	\pause
	Declarative programming language (like Lisp, Haskell, \dots)
	\begin{itemize}
		\pause
		\item Describes the logic of the program, not its control flow
		\pause
		\item The machine has not state
	\end{itemize}
\end{frame}

\begin{frame}{\subDec}
	\begin{exampleblock}{Example}
		\vspace{-1ex}
		\begin{multicols}{2}
			\texttt{int y = 0;}\\
			\texttt{int x1 = 1;}\\
			\texttt{int x2 = 2;}\\
			\texttt{void f(int z) \{}\\
				\quad \texttt{y = z + x2;}\\
			\texttt{\}}\\
			\texttt{void main() \{}\\
				\quad \texttt{f(x1);}\\
				\quad \texttt{print(y);}\\
			\texttt{\}}
			\columnbreak
			\vfill
			\[ (\lambda x_1^\mathbb{Z} x_2^\mathbb{Z}. (\lambda z^\mathbb{Z}. z + x_2) x_1) \; 1 \; 2 \]
		\end{multicols}
	\end{exampleblock}
\end{frame}

\begin{frame}[t]{\subLambda}
	\begin{block}{The language $\Lambda$}
		$\Lambda$ includes:
		\begin{enumerate}
			\item countably many variables: \quad $\{ x_1, x_2, \dots \} = V \;\subseteq\; \Lambda$
			\pause
			\item applications: \quad $(MN) \in \Lambda$, \quad for all $M, N \in \Lambda$
			\item abstractions: \quad $(\lambda x. M) \in \Lambda$, \quad for all $x \in V$ and $M \in \Lambda$
		\end{enumerate}
		\vspace{1ex}
	\end{block}
	\pause
	\begin{exampleblock}{Example}
		\begin{itemize}
			\item $x$ is just a $\lambda$-term.
			\item $\lambda x. y$ is a function which returns the constant $\lambda$-term $y$.
			\item $\lambda x. x$ is the identity function.
		\end{itemize}
	\end{exampleblock}
\end{frame}

\begin{frame}[t]{\subLambda}
	$\lambda x. M$ is just a function which takes $x$ and returns $M$.
	
	\pause
	Notice that $M$ is an arbitrary term, probably depending on $x$.
	\pause
	\begin{exampleblock}{Example}
		\vspace{-1ex}
		\begin{multicols}{2}
			\texttt{def f(x):}\\
				\quad \texttt{return x}
			\columnbreak
			\[ \lambda x. x \]
		\end{multicols}
		\pause
		\begin{multicols}{2}
			\texttt{def projection1(x, y):}\\
				\quad \texttt{return x}
			\columnbreak
			\[ \lambda x y. x  \quad = \;  \lambda x. (\lambda y. x)\]
		\end{multicols}
		\pause
		\begin{multicols}{2}
			\texttt{def comp(f, g, x):}\\
				\quad \texttt{return f(g(x))}
			\columnbreak
			\[ \lambda f g x. f (g x) \quad = \; \lambda f. (\lambda g. (\lambda x. f (g x))) \]
		\end{multicols}
	\end{exampleblock}
\end{frame}

\begin{frame}{\subLambda}
	\begin{block}{Computation}
		\[ \dots (\lambda x. M) N \dots \;\to_\beta\; \dots M [x := N] \dots \]
		\pause
		\begin{itemize}
			\item $(\lambda x. M) N$ is called a \emph{redex}.
			\pause
			\item Transforming a redex, where ever in a term, is called a \emph{one-step $\beta$-reduction}.
			\pause
			\item A term without any redexes is in \emph{normal form}.
		\end{itemize}
		\vspace{1ex}
	\end{block}
\end{frame}

\begin{frame}{\subLambda}
	Names for some common terms: \; $I = \lambda x. x, \; K = \lambda x y. x$
	\pause
	\begin{exampleblock}{Example}
		\begin{itemize}
			\item $I y = (\lambda x. x) y \to_\beta y$
			\pause
			\item $K w z = (\lambda x y. x) w z \to_\beta w$
			\pause
			\item $K (I y) z \to_\beta K y z \to_\beta y$
			\pause
			\item $K (I y) z \to_\beta I y \to_\beta y$
		\end{itemize}
	\end{exampleblock}
	\pause
	There could be more than one way to reach a normal form, but all lead to one. \pause (Church-Rosser Property)

\end{frame}

\begin{frame}[t]{\subLambda}
	Do all terms have a normal form?
	\pause
	\begin{exampleblock}{Example}
		\begin{itemize}
			\item $(\lambda x. x x)(\lambda x. x x) \to_\beta$ \pause $(\lambda x. x x)(\lambda x. x x) \to_\beta$ \pause $(\lambda x. x x)(\lambda x. x x) \to_\beta \dots$
		\end{itemize}
	\end{exampleblock}
	\pause
	This is a program that never halts.

	\pause
	\begin{block}{Can we write \emph{all} programs in $\Lambda$?}
		Actually, we can construct all partially recursive functions in $\Lambda$.\\ So $\Lambda$ is a \emph{Turing-complete} programming language.
	\end{block}
	
	\pause
	But is it really ok to apply any term on any term?
\end{frame}


\section{\secType}

\begin{frame}[t]{\subSTLC}
	We can assign types to terms, so that every term is only applicable on the terms of the proper type.
	\pause
	This will reduce our expressivity.
	\pause
	\begin{block}{The language $\Lambda_\to$}
		The set of types of $\Lambda_\to$, which is called $\mathbb{T}$ includes:
		\begin{enumerate}
			\item[4] countably many type variables: \quad $\{\alpha_1, \alpha_2, \dots\} = \mathbb{V} \;\subseteq\; \mathbb{T}$
			\pause
			\item[5] function types: \quad $\sigma \to \tau \in \mathbb{T}$, \quad for all $\sigma, \tau \in \mathbb{T}$
		\end{enumerate}
		\pause
		We state that the term $M$ is of the type $\sigma$ by
		\[M : \sigma\]
	\end{block}
	\pause
	We want to let the term \; $M N$ \; to be \emph{typable}, only if \; $M : \sigma \to \tau$ \; and \; $N : \sigma$.
\end{frame}

\begin{frame}{\subSTLC}
	\begin{exampleblock}{Example}
		\begin{itemize}
			\item $I_\sigma = \lambda x^\sigma . x : \sigma \to \sigma$
			\pause
			\item $K_{\sigma,\tau} = \lambda x^\sigma y^\tau. x : \sigma \to (\tau \to \sigma)$
			\pause
			\item $S_{\sigma,\tau,\delta} = \lambda \; g~^{\sigma \to (\tau \to \delta)} \; f~^{\sigma \to \tau} \; x~^\sigma. g x (f x) :$\\ \quad $(\sigma \to (\tau \to \delta)) \to ((\sigma \to \tau) \to (\sigma \to \delta))$
		\end{itemize}
	\end{exampleblock}
	\pause
	It is also possible to state type of a term in a specific \emph{context}.
	
	\pause
	A \emph{context} is a sequence of typed term variables.
	\[\Gamma = x_1^{\sigma_1}, x_2^{\sigma_2}, \dots, x_n^{\sigma_n}\]
\end{frame}

\begin{frame}[t]{\subTypability}
	We define inductively what it means for a term $M$ to be typable of type $\sigma$, in a context $\Gamma$, denoted by $\Gamma \vdash M : \sigma$.
	
	\pause
	First, we need to limit applications:
	\begin{prooftree}
		\AXC{$\Gamma \vdash M : \sigma \to \tau$}
		\AXC{$\Gamma \vdash N : \sigma$}
		\RightLabel{app}
		\BIC{$\Gamma \vdash M N : \tau$}
	\end{prooftree}
	\pause
	Second, we need to tell how to construct a function:
	\begin{prooftree}
		\AXC{$\Gamma, x^\sigma \vdash M : \tau$}
		\RightLabel{abs}
		\UIC{$\Gamma \vdash \lambda x^\sigma. M : \sigma \to \tau$}
	\end{prooftree}
	\pause
	And at last, we need a basis for our inductive definition: $(x^\sigma \not \in \Gamma)$
	\begin{prooftree}
		\AXC{}
		\RightLabel{var}
		\UIC{$\Gamma, x^\sigma \vdash x : \sigma$}
	\end{prooftree}
\end{frame}

\begin{frame}[t]{\subTypability}
	\quad
		{\small
		\begin{prooftree}
			\hspace*{-7mm}
			\AXC{}
			\RightLabel{var}
			\UIC{$\Gamma \vdash g : \sigma \to (\tau \to \delta)$}
			\AXC{}
			\RightLabel{var}
			\UIC{$\Gamma \vdash x : \sigma$}
			\RightLabel{app}
			\BIC{$\Gamma \vdash g x : \tau \to \delta$}
			\AXC{}
			\RightLabel{var}
			\UIC{$\Gamma \vdash f : \sigma \to \tau$}
			\AXC{}
			\RightLabel{var}
			\UIC{$\Gamma \vdash x : \sigma$}
			\RightLabel{app}
			\BIC{$\Gamma \vdash f x : \tau$}
			\RightLabel{app}
			\BIC{${\color{red} \Gamma = \quad} g~^{\sigma \to (\tau \to \delta)}, \; f~^{\sigma \to \tau}, \; x~^\sigma \; \vdash \; g x (f x) : \delta$}
			\RightLabel{abs}
			\UIC{$g~^{\sigma \to (\tau \to \delta)}, \; f~^{\sigma \to \tau} \; \vdash \; \lambda x~^\sigma. g x (f x) :\sigma \to \delta$}
			\RightLabel{abs}
			\UIC{$g~^{\sigma \to (\tau \to \delta)} \; \vdash \; \lambda f~^{\sigma \to \tau} \; x~^\sigma. g x (f x) : (\sigma \to \tau) \to (\sigma \to \delta)$}
			\RightLabel{abs}
			\UIC{$\vdash \lambda \; g~^{\sigma \to (\tau \to \delta)} \; f~^{\sigma \to \tau} \; x~^\sigma. g x (f x) : (\sigma \to (\tau \to \delta)) \to ((\sigma \to \tau) \to (\sigma \to \delta))$}
		\end{prooftree}
		}%\small
		\vfill
		\pause
		Does this ring a bell?
\end{frame}

\section{\secTruth}

\begin{frame}{\subTruth}
	\begin{block}{Law of Excluded Middle}
		\[ P \vee \neg P \]
		\pause
		Who would disagree? \pause Intuitionists!
		\vspace{1ex}
	\end{block}
	\pause
	But isn't evert statement either \emph{true} or \emph{not true}? \\
	\pause
	Depends on what do we mean by ``truth''.
\end{frame}

\begin{frame}[t]{\subTruth}
Intuitionists believe that truth of a statement is a mental construction.

\pause
If we take everything either true or not true, we would lose data!

\pause
\begin{exampleblock}{Example}
	\textit{Question.}\; Are there two irrationals $a$ and $b$ where $a^b$ is rational?

	\textit{Answer.}\; Yes. We know that $\sqrt{2}^{\sqrt{2}}$ is either rational or irrational. In the first case, take $a = \sqrt{2}$ and $b = \sqrt{2}$. In the second case, take $a = \sqrt{2}^{\sqrt{2}}$ and $b = \sqrt{2}$.
	\vspace{1ex}
\end{exampleblock}

\pause
But what are $a$ and $b$ actually?
\end{frame}


\begin{frame}[t]{\subBHK}
 \emph{The proof} of a statement is the construction that makes it true.

\pause
\begin{block}{The BHK Proof Interpretation}
	A proof for
	\begin{itemize}
		\item $A \wedge B$ is a construction made of a proof  for $A$ and a proof for $B$.
		\item $A \vee B$ is a construction made of a proof  for either of $A$ and $B$, and a way to tell which one.
		\item $A \to B$ is a construction that takes any proof for $A$ and constructs a proof for $B$.
		\item $\bot$ (the \emph{false} statement) is non existent!
	\end{itemize}
	$\neg A$ is just an abbreviation for $A \to \bot$.
	\vspace{1ex}
\end{block}
\end{frame}

\begin{frame}{\subIntLog}
	Recall that \emph{natural deduction} is designed so that it fulfils what the BHK interpretation expects of a proof:
	{\footnotesize
	\begin{prooftree}
		\AXC{}
		\RightLabel{$Id$}
		\UIC{$\Gamma, \sigma \vdash \sigma$}
		\AXC{$\Gamma \vdash \bot$}
		\RightLabel{$\bot E$}
		\UIC{$\Gamma \vdash \sigma$}
		\noLine\BIC{}
	\end{prooftree}
	\begin{prooftree}
		\AXC{$\Gamma , \sigma \vdash \tau$}
		\RightLabel{$\to I$}
		\UIC{$\Gamma \vdash \sigma \to \tau$}
		\AXC{$\Gamma \vdash \sigma \to \tau$}
		\AXC{$ \Gamma \vdash \sigma$}
		\RightLabel{$\to E$}
		\BIC{$\Gamma \vdash \tau$}
		\noLine\BIC{}
	\end{prooftree}
	\begin{prooftree}
		\AXC{$\Gamma \vdash \sigma$}
		\AXC{$ \Gamma \vdash \tau$}
		\RightLabel{$\wedge I$}
		\BIC{$\Gamma \vdash \sigma \land \tau$}
		\AXC{$\Gamma \vdash \sigma \land \tau$}
		\RightLabel{$\wedge E _1$}
		\UIC{$\Gamma \vdash \sigma$}
		\AXC{$\Gamma \vdash \sigma \land \tau$}
		\RightLabel{$\wedge E _2$}
		\UIC{$\Gamma \vdash \tau$}
		\noLine\TIC{}
	\end{prooftree}
	\begin{prooftree}
		\AXC{$\Gamma \vdash \sigma$}
		\RightLabel{$\vee I _1$}
		\UIC{$\Gamma \vdash \sigma \lor \tau$}
		\AXC{$\Gamma \vdash \tau$}
		\RightLabel{$\vee I _2$}
		\UIC{$\Gamma \vdash \sigma \lor \tau$}
		\AXC{$\Gamma, \sigma\vdash \delta$}
		\AXC{$ \Gamma, \tau \vdash \delta$}
		\RightLabel{$\vee E$}
		\BIC{$\Gamma, \sigma \vee \tau \vdash \delta$}
		\noLine\TIC{}
	\end{prooftree}
	}%\footnotesize
\end{frame}

\begin{frame}{\subIntLog}
	Take the implicational fragment of the propositional logic.
	\begin{block}{Natural Deduction for Implicational Fragment of Intuitionistic Logic}
		{\small

			\begin{prooftree}
				\AXC{$\Gamma \vdash \sigma \to 
				\tau$}
				\AXC{$\Gamma \vdash \sigma$}
				\RightLabel{$\to E$}
				\BIC{$\Gamma \vdash \tau$}

				\AXC{$\Gamma, \sigma \vdash \tau$}
				\RightLabel{$\to I$}
				\UIC{$\Gamma \vdash \sigma \to \tau$}

				\AXC{}
				\RightLabel{$Id$}
				\UIC{$\Gamma, \sigma \vdash \sigma$}
				
				\noLine\TIC{}
			\end{prooftree}
		}%\small
	\end{block}
	\pause
	If we add the \emph{proof-terms} to this system, then \emph{provability} in implicational natural deduction would be just the same as \emph{typability} in $\Lambda_\to$.
\end{frame}

\begin{frame}{\subIntLog}
	$\lambda$-calculus and logic are essentially the same, modulo the proof-terms.

	So we can see
		\begin{center}
			propositions as types, and\\
			their proofs as the terms of their corresponding type.
		\end{center}

	This is called the \emph{Curry-Howard correspondence}.
	\begin{block}{Typability Relation in $\Lambda_\to$}
		{\footnotesize
			\begin{prooftree}
				\AXC{$\Gamma \vdash M : \sigma \to \tau$}
				\AXC{$\Gamma \vdash N : \sigma$}
				\RightLabel{app}
				\BIC{$\Gamma \vdash M N : \tau$}

				\AXC{$\Gamma, x^\sigma \vdash M : \tau$}
				\RightLabel{abs}
				\UIC{$\Gamma \vdash \lambda x^\sigma. M : \sigma \to \tau$}

				\AXC{}
				\UIC{$\Gamma, x^\sigma \vdash x : \sigma$}
				
				\noLine\TIC{}
			\end{prooftree}
		}%\small
	\end{block}
\end{frame}

\begin{frame}{\subIntLog}
	Define the \emph{enriched} simply typed lambda calculus $\Lambda_{\text{IPC}}$ as follows:
	{\footnotesize
	\begin{prooftree}
		\AXC{}
		\RightLabel{var}
		\UIC{$\Gamma, x^\sigma \vdash x : \sigma$}
		\AXC{$\Gamma \vdash \text{contradiction} : \bot$}
		\RightLabel{$\bot E$}
		\UIC{$\Gamma \vdash M : \sigma$}
		\noLine\BIC{}
	\end{prooftree}
	\begin{prooftree}
		\AXC{$\Gamma, x^\sigma \vdash M : \tau$}
		\RightLabel{$\to I$}
		\UIC{$\Gamma \vdash \lambda x^\sigma . M : \sigma \to \tau$}
		\AXC{$\Gamma \vdash M : \sigma \to \tau$}
		\AXC{$ \Gamma \vdash N : \sigma$}
		\RightLabel{$\to E$}
		\BIC{$\Gamma \vdash M N : \tau$}
		\noLine\BIC{}
	\end{prooftree}
	\begin{prooftree}
		\AXC{$\Gamma \vdash M : \sigma$}
		\AXC{$ \Gamma \vdash N : \tau$}
		\RightLabel{$\wedge I$}
		\BIC{$\Gamma \vdash \langle M, N \rangle : \sigma \land \tau$}
		\AXC{$\Gamma \vdash M : \sigma \land \tau$}
		\RightLabel{$\wedge E _1$}
		\UIC{$\Gamma \vdash \pi_1(M) : \sigma$}
		\AXC{$\Gamma \vdash M : \sigma \land \tau$}
		\RightLabel{$\wedge E _2$}
		\UIC{$\Gamma \vdash \pi_2(M) : \tau$}
		\noLine\TIC{}
	\end{prooftree}
	\begin{prooftree}
		\hspace*{-6mm}
		\AXC{$\Gamma \vdash M : \sigma$}
		\RightLabel{$\vee I _1$}
		\UIC{$\Gamma \vdash \langle 0, M \rangle : \sigma \lor \tau$}
		\AXC{$\Gamma \vdash M : \tau$}
		\RightLabel{$\vee I _2$}
		\UIC{$\Gamma \vdash \langle 1, M \rangle : \sigma \lor \tau$}
		\AXC{$\Gamma, x^\sigma \vdash M : \delta$}
		\AXC{$ \Gamma, y^\tau \vdash N : \delta$}
		\RightLabel{$\vee E$}
		\BIC{$\Gamma, z^{\sigma \vee \tau} \vdash \text{if} \; z \; \text{then} \; M \; \text{else} \; N : \delta$}
		\noLine\TIC{}
	\end{prooftree}
	}%\footnotesize
\end{frame}

\begin{frame}{\subIntLog}
	So we are asked to prove something a $\lambda$-calculus, we can just code the proof!
	\pause
	\begin{exampleblock}{Example}
		\emph{Question. }~ Prove $P \to P$ in $\Lambda_\text{IPC}$.\\
		\pause
		\emph{Answer. }~ $\lambda x^P . x : P \to P$, or the identity function over the type $P$, i.e. $I_P$.
	\end{exampleblock}
	\pause
	\begin{exampleblock}{Example}
		\emph{Question. }~ Prove $P \to (Q \to P)$ in $\Lambda_\text{IPC}$.\\
		\pause
		\emph{Answer. }~ $\lambda x^P y^Q . x : P \to (Q \to P)$, or the left projection function over the types $P$ and $Q$, i.e. $K_{P,Q}$.
	\end{exampleblock}
\end{frame}

\begin{frame}{\subIntLog}
	Recall that $\neg P$ is just $P \to \bot$.
	\begin{exampleblock}{Example}
		\emph{Question. }~ Prove $P \to \neg \neg P$ in $\Lambda_\text{IPC}$.\\
		\pause
		\emph{Answer. }\quad
		\[ \lambda x^P f~^{P \to \bot} . f x : P \to ((P \to \bot) \to \bot) \]
	\end{exampleblock}
	\pause
	But can we prove $\neg \neg P \to P$? Or $P \vee \neg P$?
\end{frame}

\begin{frame}{}
	What else can be expressed in $\lambda$-calculus?
\end{frame}

\section{\secDep}

\begin{frame}{\subDepFun}
	Suppose we have a way to \emph{index} a family of types by another type.
	
	\pause
	For example, let $\mathbb{N}$ be the type of \emph{natural numbers}. So $n : \mathbb{N}$ means that $n$ is a natural number. Now suppose $P$ is a family of types indexed by terms of $\mathbb{N}$.

	\pause
	Then w $P n$ states the claim that the 
	\begin{block}{The BHK Interpretation, Revised}
		A proof for
		\begin{itemize}
			\item $\forall x : A, B x$ is a construction which takes any term $x$ of $A$ to a proof for 
		\end{itemize}
	\end{block}
\end{frame}

\begin{frame}{\subDepFun}
	\begin{block}{Dependent Type Theory}
		\begin{prooftree}
			\AXC{$\Gamma \vdash M : \forall x : A, B x$}
			\AXC{$\Gamma \vdash N : A$}
			\RightLabel{app}
			\BIC{$\Gamma \vdash M N : B M$}

			\AXC{$\Gamma, x^A \vdash M : B x$}
			\RightLabel{abs}
			\UIC{$\Gamma \vdash \lambda x^A. M : \forall x : A, B x$}
			\noLine\BIC{}
		\end{prooftree}
	\end{block}

	\pause
	Notice that $A \to B$ is just $\forall \_ : A, B \_$.
\end{frame}

\bibliographystyle{apalike}
\bibliography{references}

\begin{frame}[noframenumbering]{\quad}
	\begin{center}
		\Huge Thank you!
	\end{center}
\end{frame}

\end{document}
